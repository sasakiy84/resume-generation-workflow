\hypertarget{ux524dux56deux306eux8981ux7d04}{%
\section{前回の要約}\label{ux524dux56deux306eux8981ux7d04}}

\hypertarget{ux5352ux8ad6ux306bux3064ux3044ux3066ux306eux30d2ux30a2ux30eaux30f3ux30b0}{%
\subsection{卒論についてのヒアリング}\label{ux5352ux8ad6ux306bux3064ux3044ux3066ux306eux30d2ux30a2ux30eaux30f3ux30b0}}

以下のことを佐々木から伝え、それに合わせて TODO を決めた\cite{ChongShi2011}

\begin{itemize}
\tightlist
\item
  東京大学図書館情報学研究室に進学したい
\item
  院にいってきちんと研究ができるレベルへ向けて指導をお願いしたい
\item
  方法論からではなく対象から研究テーマを絞りたい。浅石先生の教科書についての分析のようなものをイメージしている
\item
  卒論で身につけたいスキルは具体的に定まっていない
\end{itemize}

\hypertarget{ux5f71ux6d66ux5148ux751fux304cux81eaux8eabux306eux7814ux7a76ux9818ux57dfux306bux3064ux3044ux3066ux3069ux3046ux8003ux3048ux3066ux3044ux308bux304b}{%
\subsection{影浦先生が自身の研究領域についてどう考えているか}\label{ux5f71ux6d66ux5148ux751fux304cux81eaux8eabux306eux7814ux7a76ux9818ux57dfux306bux3064ux3044ux3066ux3069ux3046ux8003ux3048ux3066ux3044ux308bux304b}}

\begin{itemize}
\tightlist
\item
  影浦先生の研究領域はわかりやすいかたちで提示することができず、院に来るとすれば問いの共有について苦労するだろう
\item
  佐々木がなぜ影浦先生を指導教員に選んだのか、言葉にできるとよい。また、影浦先生のもとで学んでなにができるようになったかを言語化できるようにするのが影浦先生側の目標である
\item
  言語というものが透明な媒体でなくなるときがあり、明示的に操作しなければいけないときがある
\item
  言語学とどう違うかという点に関しては必要になったら話す
\end{itemize}

\hypertarget{todo}{%
\section{TODO}\label{todo}}

\hypertarget{ux7814ux7a76ux5bfeux8c61}{%
\subsection{研究対象}\label{ux7814ux7a76ux5bfeux8c61}}

カテゴリのレベルで対象を5つあげる。\\
また、カテゴリのレベルに対応するデータのレベルで、対象が一意に定まらない場合は、候補を二つ以上あげる。

そして、カテゴリとデータのレベルそれぞれについて以下のことを書き出す

\begin{itemize}
\tightlist
\item
  特徴
\item
  なにをどのように明らかにするか
\item
  明らかになったことはどのような意味を持つか
\item
  それを対象として研究をすすめたときのメリット・デメリット

  \begin{itemize}
  \tightlist
  \item
    内的ポジティブなもの、内的ネガティブなもの、外的ポジティブなもの、外的ネガティブなものの順番を意識して考える
  \item
    自身が身につけたいスキルとの関連でもメリットとデメリットを考慮する
  \end{itemize}
\end{itemize}

\hypertarget{ux8eabux306bux3064ux3051ux305fux3044ux30b9ux30adux30eb}{%
\subsection{身につけたいスキル}\label{ux8eabux306bux3064ux3051ux305fux3044ux30b9ux30adux30eb}}

卒論執筆を通して身につけたいスキルを 3\textasciitilde7
個程度挙げる。例えば、以下のようなレベル

\begin{itemize}
\tightlist
\item
  Transformer を使ってみる
\item
  Julia を学んでみる
\item
  新しい言語(たとえば中国語)を身につける
\item
  何らかの統計的な分析のフレームワークを身につける
\end{itemize}

\hypertarget{ux7814ux7a76ux5bfeux8c61ux306bux3064ux3044ux3066}{%
\section{研究対象について}\label{ux7814ux7a76ux5bfeux8c61ux306bux3064ux3044ux3066}}

以下の五つのカテゴリを設定した。

\begin{itemize}
\tightlist
\item
  国会議事録
\item
  プログラミング言語の仕様書とチュートリアル
\item
  英語と日本語のプログラミングチュートリアル
\item
  国立国会図書館の蔵書構成
\item
  動画教材と文字教材
\end{itemize}

以下の点はできなかった

\begin{itemize}
\tightlist
\item
  国会議事録と国立国会図書館の蔵書構成について、二つ以上のデータをあげること
\item
  それぞれの対象について、先行研究の網羅的な収集(ネットで関連がありそうなキーワードを検索し、上位十数件を確認するくらいしかできていない)
\end{itemize}

\hypertarget{ux56fdux4f1aux8b70ux4e8bux9332}{%
\subsection{国会議事録}\label{ux56fdux4f1aux8b70ux4e8bux9332}}

\hypertarget{ux30abux30c6ux30b4ux30eaux306eux7279ux5fb4}{%
\subsubsection{カテゴリの特徴}\label{ux30abux30c6ux30b4ux30eaux306eux7279ux5fb4}}

\begin{itemize}
\tightlist
\item
  国権の最高機関である国会でのやり取りを文字起こししたテキストデータであり、政治研究等に活用されている
\item
  発言者と発言内容の組が最小単位であり、それを会議ごとや会期単位で検索できる
\item
  機械的に文字起こしされている部分がある
\item
  第一回国会からのデータが網羅的にそろっている
\end{itemize}

\hypertarget{ux30abux30c6ux30b4ux30eaux306bux5bfeux3057ux3066ux3069ux306eux3088ux3046ux306aux65b9ux6cd5ux3067ux306aux306bux3092ux660eux3089ux304bux306bux3059ux308bux304b}{%
\subsubsection{カテゴリに対してどのような方法でなにを明らかにするか}\label{ux30abux30c6ux30b4ux30eaux306bux5bfeux3057ux3066ux3069ux306eux3088ux3046ux306aux65b9ux6cd5ux3067ux306aux306bux3092ux660eux3089ux304bux306bux3059ux308bux304b}}

テキスト特徴量による記述を行う。また、議事録の特徴である発言者のメタデータを活かして、発言者ごと、所属会派ごと、会議種別ごと、会期ごとの分析も行う。さらに、時系列データであるため、経時的な変化についても着目する。これらによって、国会議事録の量的な特徴を明らかにする。\\
また、国政と関連した記録であることから、日本の政治学・行政学的な視点と計量データを結び付けて議論する。これによって、日本において発生した事象が国会議事に反映されているのか、またされているとしたらどのように反映されているのかを明らかにする。

\hypertarget{ux30abux30c6ux30b4ux30eaux306bux5bfeux3057ux3066ux306eux30e1ux30eaux30c3ux30c8ux3068ux30c7ux30e1ux30eaux30c3ux30c8}{%
\subsubsection{カテゴリに対してのメリットとデメリット}\label{ux30abux30c6ux30b4ux30eaux306bux5bfeux3057ux3066ux306eux30e1ux30eaux30c3ux30c8ux3068ux30c7ux30e1ux30eaux30c3ux30c8}}

\begin{itemize}
\tightlist
\item
  全文データである
\item
  発言者の立場について、ある程度のメタ情報がそろっている
\item
  音声でのやり取りを書き起こしたものであるので、実際のやりとりと比べると情報の欠落がある
\item
  国会の審議だけで国政が行われているわけではない
\item
  誤字脱字が考えられる
\item
  国会議事録に対する分析手法を試すことで、地方議会議事録などの分析にも応用できる可能性がある
\item
  国会議事録に対する分析手法を試すことで、国民が国政を評価する際の判断材料になる
\item
  国会議事録を用いた分析はある程度先行研究がある
\item
  テキスト処理の技術が身につけられる
\item
  日本政治に対する知識が身につけられる
\end{itemize}

\hypertarget{ux30c7ux30fcux30bf-1}{%
\subsubsection{データ 1}\label{ux30c7ux30fcux30bf-1}}

法律案ごとに分類した国会議事録データ

\hypertarget{ux30c7ux30fcux30bfux306eux7279ux5fb4}{%
\paragraph{データの特徴}\label{ux30c7ux30fcux30bfux306eux7279ux5fb4}}

\href{https://hourei.ndl.go.jp/\#/}{日本法令索引}において検索可能な国会の法律案の審議経過データを対象にする\\
対象になる法律のデータの件数は以下の通り

\begin{longtable}[]{@{}ll@{}}
\toprule\noalign{}
種別 & 件数 \\
\midrule\noalign{}
\endhead
\bottomrule\noalign{}
\endlastfoot
閣法 & 10332 \\
衆法 & 4225 \\
参法 & 2058 \\
\end{longtable}

それぞれの法律に対して、以下の情報が取得できる(\href{https://hourei.ndl.go.jp/\#/detail?lawId=0000039225\&searchDiv=2\&current=1}{参考にしたページ}
)

\begin{itemize}
\tightlist
\item
  交付年月日
\item
  法律の分類
\item
  提出回次
\item
  種別
\item
  審議経過が含まれる国会の議事録

  \begin{itemize}
  \tightlist
  \item
    会議種別(xxx 委員会/本会議)
  \item
    会議日時
  \item
    開始ページ
  \item
    終了ページ
  \item
    審議経過種別(趣旨説明、質疑、討論、採決など)
  \end{itemize}
\end{itemize}

\hypertarget{ux30c7ux30fcux30bfux306bux5bfeux3057ux3066ux3069ux306eux3088ux3046ux306aux65b9ux6cd5ux3067ux306aux306bux3092ux660eux3089ux304bux306bux3059ux308bux304b}{%
\paragraph{データに対してどのような方法でなにを明らかにするか}\label{ux30c7ux30fcux30bfux306bux5bfeux3057ux3066ux3069ux306eux3088ux3046ux306aux65b9ux6cd5ux3067ux306aux306bux3092ux660eux3089ux304bux306bux3059ux308bux304b}}

法律案に関する発言のまとまりに対して、テキスト特徴量による比較を行う。また、法律案の審議について、経時的な変化があるかを確認する。これらによって、日本の立法機関において、そもそも何がよく議論されているのか、そしてどのように議論されているのかを明らかにする。\\
さらに、災害時の会期について、関連する法案がどのように審議されているかを計量的なデータを使って確認することで、国会が突発的な事態にどのように対応しているのかを分析する。

\hypertarget{ux30c7ux30fcux30bfux306bux5bfeux3057ux3066ux306eux30e1ux30eaux30c3ux30c8ux3068ux30c7ux30e1ux30eaux30c3ux30c8}{%
\paragraph{データに対してのメリットとデメリット}\label{ux30c7ux30fcux30bfux306bux5bfeux3057ux3066ux306eux30e1ux30eaux30c3ux30c8ux3068ux30c7ux30e1ux30eaux30c3ux30c8}}

\begin{itemize}
\tightlist
\item
  国会での議事における法律制定過程をすべての法律について分析できる
\item
  国会で注目された法律を定量的に把握できる
\item
  法律の制定過程で何人の議員が議論に参加し、どれくらい発言したのかを定量的に把握できる
\item
  他の法律とまとめて審議されている法律があるため、機械的に分析すると誤差が生じる
\end{itemize}

\hypertarget{ux30d7ux30edux30b0ux30e9ux30dfux30f3ux30b0ux8a00ux8a9eux306eux4ed5ux69d8ux66f8ux3068ux30c1ux30e5ux30fcux30c8ux30eaux30a2ux30eb}{%
\subsection{プログラミング言語の仕様書とチュートリアル}\label{ux30d7ux30edux30b0ux30e9ux30dfux30f3ux30b0ux8a00ux8a9eux306eux4ed5ux69d8ux66f8ux3068ux30c1ux30e5ux30fcux30c8ux30eaux30a2ux30eb}}

\hypertarget{ux30abux30c6ux30b4ux30eaux306eux7279ux5fb4-1}{%
\subsubsection{カテゴリの特徴}\label{ux30abux30c6ux30b4ux30eaux306eux7279ux5fb4-1}}

プログラミング言語の仕様書の特徴

\begin{itemize}
\tightlist
\item
  人工言語についての仕様書であり、自然言語と異なり曖昧さがほとんどない
\item
  記号列の表記方法と意味を定義しており、言語自体を実装する人か言語について深い知識を得たい人向けである
\item
  理論的な概念に基づいているものが多い
\item
  多くの種類がある
\end{itemize}

プログラミング言語のチュートリアルの特徴

\begin{itemize}
\tightlist
\item
  仕様書によって実装された言語の使い方を、学習者のために記述している
\item
  一つの言語に対して、公式が作成したものから素人が書いたものまで数多くのチュートリアルがある
\end{itemize}

\hypertarget{ux30abux30c6ux30b4ux30eaux306bux5bfeux3057ux3066ux3069ux306eux3088ux3046ux306aux65b9ux6cd5ux3067ux306aux306bux3092ux660eux3089ux304bux306bux3059ux308bux304b-1}{%
\subsubsection{カテゴリに対してどのような方法でなにを明らかにするか}\label{ux30abux30c6ux30b4ux30eaux306bux5bfeux3057ux3066ux3069ux306eux3088ux3046ux306aux65b9ux6cd5ux3067ux306aux306bux3092ux660eux3089ux304bux306bux3059ux308bux304b-1}}

網羅的な仕様書から、学習用のチュートリアルに変換されるときに、知識はどの程度圧縮され、どの部分が削られるのか、あるいはどのような説明が付け加えられるのかを分析する。\\
これによって、プログラミング言語を使えるようになるための必要と考えられている知識や、最低限の使用には不必要であると考えられている知識の種類や、その割合を明らかにする

\hypertarget{ux30abux30c6ux30b4ux30eaux306bux5bfeux3057ux3066ux306eux30e1ux30eaux30c3ux30c8ux3068ux30c7ux30e1ux30eaux30c3ux30c8-1}{%
\subsubsection{カテゴリに対してのメリットとデメリット}\label{ux30abux30c6ux30b4ux30eaux306bux5bfeux3057ux3066ux306eux30e1ux30eaux30c3ux30c8ux3068ux30c7ux30e1ux30eaux30c3ux30c8-1}}

\begin{itemize}
\tightlist
\item
  人工言語であるため、知識の総体がある意味で明確化されている
\item
  言語仕様だけでは、背景となっている理論的な部分まで分析できない
\item
  近年、プログラミングは世間で注目されている
\item
  研究するうえでプログラミングに対する理論的な知識を身につけることができる
\end{itemize}

\hypertarget{ux30c7ux30fcux30bf-1-1}{%
\subsubsection{データ 1}\label{ux30c7ux30fcux30bf-1-1}}

\href{https://html.spec.whatwg.org/multipage/\#toc-index}{HTML Living
Standard} と
\href{https://developer.mozilla.org/en-US/docs/Learn/HTML}{MDN HTML}
の比較

\hypertarget{ux30c7ux30fcux30bfux306eux7279ux5fb4-1}{%
\paragraph{データの特徴}\label{ux30c7ux30fcux30bfux306eux7279ux5fb4-1}}

HTML Living Standard の特徴

\begin{itemize}
\tightlist
\item
  HTML の仕様が定義されている。また、例示なども行っている
\item
  RFC などで定義されている複数の概念に依存している(HTTP や DOM など)
\item
  現実の実装が複数存在している
\item
  HTML で公開されている
\item
  キーワードには定義箇所へのリンクが貼ってある
\item
  ブラウザの実装者向けであると同時に、ブラウザを利用したアプリケーションを実装する人向けの情報も書かれている
\end{itemize}

MDN HTML の特徴

\begin{itemize}
\tightlist
\item
  HTML Living Standard  に沿って Firefox を実装している Mozilla
  が運営している
\item
  ウェブアプリケーションを実装するコミュニティのなかで信頼されている情報源である
\item
  HTML を初めて学ぶ人を対象としたチュートリアルである
\item
  HTML で公開されている
\item
  キーワードには説明へのリンクが貼ってある
\end{itemize}

\hypertarget{ux30c7ux30fcux30bfux306bux5bfeux3057ux3066ux3069ux306eux3088ux3046ux306aux65b9ux6cd5ux3067ux306aux306bux3092ux660eux3089ux304bux306bux3059ux308bux304b-1}{%
\paragraph{データに対してどのような方法でなにを明らかにするか}\label{ux30c7ux30fcux30bfux306bux5bfeux3057ux3066ux3069ux306eux3088ux3046ux306aux65b9ux6cd5ux3067ux306aux306bux3092ux660eux3089ux304bux306bux3059ux308bux304b-1}}

それぞれの文書に対して、単語数、見出しの数、キーワードの数などのテキスト特徴量を計量的に分析することを通して、全体像を把握する。\\
これにより、知識がどの程度圧縮されたかや、どのようなキーワードが削られたのかを確認する。\\
次に、特定の機能についての記述を抜き出して、表現を具体的に観察する。具体的には、両者で説明されている
image や table についての記述を想定している。\\
これにより、具体的な表現のレベルで仕様書とチュートリアルの違いを明らかにする。特に、仕様書の説明とくらべてチュートリアルで追加されるであろう例示などを把握することで、学習用のコンテンツとしてどのような特徴をもっているかを明らかにする

\hypertarget{ux30c7ux30fcux30bfux306bux5bfeux3057ux3066ux306eux30e1ux30eaux30c3ux30c8ux3068ux30c7ux30e1ux30eaux30c3ux30c8-1}{%
\paragraph{データに対してのメリットとデメリット}\label{ux30c7ux30fcux30bfux306bux5bfeux3057ux3066ux306eux30e1ux30eaux30c3ux30c8ux3068ux30c7ux30e1ux30eaux30c3ux30c8-1}}

\begin{itemize}
\tightlist
\item
  全文テキストが入手できるため、網羅性のある分析が行える
\item
  HTML
  で記述されており、構造化されているため、分析に活用できる項目が多い
\item
  HTML はプログラミング言語とはいえない(人工言語ではある)
\item
  HTML
  は世界的に使われている人工言語であり、中学の情報の教科書にも記載されている
\item
  英語に慣れることができる
\item
  HTML に詳しくなれる
\end{itemize}

\hypertarget{ux30c7ux30fcux30bf-2}{%
\subsubsection{データ 2}\label{ux30c7ux30fcux30bf-2}}

\href{https://docs.python.org/3/}{Python の公式ドキュメント}

\hypertarget{ux30c7ux30fcux30bfux306eux7279ux5fb4-2}{%
\paragraph{データの特徴}\label{ux30c7ux30fcux30bfux306eux7279ux5fb4-2}}

\begin{itemize}
\tightlist
\item
  言語仕様とチュートリアルが同じ団体によって執筆され、同じサイト上で運用されている
\item
  Python の言語仕様と、モジュールについての仕様が記載されている
\item
  Glossary
  がある(チュートリアルで出現する各単語からリンクが貼られているわけではない)
\item
  HTML で公開されている
\end{itemize}

\hypertarget{ux30c7ux30fcux30bfux306bux5bfeux3057ux3066ux3069ux306eux3088ux3046ux306aux65b9ux6cd5ux3067ux306aux306bux3092ux660eux3089ux304bux306bux3059ux308bux304b-2}{%
\paragraph{データに対してどのような方法でなにを明らかにするか}\label{ux30c7ux30fcux30bfux306bux5bfeux3057ux3066ux3069ux306eux3088ux3046ux306aux65b9ux6cd5ux3067ux306aux306bux3092ux660eux3089ux304bux306bux3059ux308bux304b-2}}

\href{https://docs.python.org/3/library/index.html}{Library
Reference}、\href{https://docs.python.org/3/reference/index.html}{Language
Reference}、\href{https://docs.python.org/3/using/index.html}{Python
Setup and
Usage}などを仕様書、\href{https://docs.python.org/3/tutorial/index.html}{Tutorial}
をチュートリアルとみなす。\\
それぞれの文書に対してテキスト特徴量を記述し、全体像を把握する。\\
これにより、仕様書からチュートリアルへの変換で、全体像がどのように変化したかを明らかにする。\\
次に、特定の機能についての記述を抜き出して、表現を具体的に観察する。具体的には、数値型などの記述を想定している。特に、仕様書の説明とくらべてチュートリアルで追加されるであろう例示などを把握する。\\
これにより、学習用のコンテンツとしてチュートリアルがどのような特徴をもっているかを明らかにする

\hypertarget{ux30c7ux30fcux30bfux306bux5bfeux3057ux3066ux306eux30e1ux30eaux30c3ux30c8ux3068ux30c7ux30e1ux30eaux30c3ux30c8-2}{%
\paragraph{データに対してのメリットとデメリット}\label{ux30c7ux30fcux30bfux306bux5bfeux3057ux3066ux306eux30e1ux30eaux30c3ux30c8ux3068ux30c7ux30e1ux30eaux30c3ux30c8-2}}

\begin{itemize}
\tightlist
\item
  全文テキストが入手できるため、網羅性のある分析が行える
\item
  HTML
  で記述されており、構造化されているため、分析に活用できる項目が多い
\item
  どこまでが仕様書かについて明示されていないため、判断が分かれる可能性がある
\item
  機械学習やデータ分析などで使われており、勢いのある言語であるため、その学習資源を評価することは重要である
\end{itemize}

\hypertarget{ux82f1ux8a9eux3068ux65e5ux672cux8a9eux306eux30d7ux30edux30b0ux30e9ux30dfux30f3ux30b0ux30c1ux30e5ux30fcux30c8ux30eaux30a2ux30eb}{%
\subsection{英語と日本語のプログラミングチュートリアル}\label{ux82f1ux8a9eux3068ux65e5ux672cux8a9eux306eux30d7ux30edux30b0ux30e9ux30dfux30f3ux30b0ux30c1ux30e5ux30fcux30c8ux30eaux30a2ux30eb}}

ここでのプログラミングチュートリアルとは、ネット上に公開されているプログラミングの基礎やフレームワーク、ライブラリを学ぶためのチュートリアルのことを想定している

\hypertarget{ux30abux30c6ux30b4ux30eaux306eux7279ux5fb4-2}{%
\subsubsection{カテゴリの特徴}\label{ux30abux30c6ux30b4ux30eaux306eux7279ux5fb4-2}}

\begin{itemize}
\tightlist
\item
  チュートリアルの充実度は管理する主体によってさまざまである
\item
  公式、あるいはそれに近い団体が作成した英語のチュートリアルを、有志が日本語に訳したものが多い
\item
  ライブラリやフレームワークを学ぶときに、第一に参照する文書である
\item
  説明対象のアップデートの頻度に合わせてチュートリアルもアップデートされる
\item
  英語の発音をそのまま訳出している事例が多い
\item
  アルファベット綴りをそのまま訳出しているものがある
\item
  ネット上の不特定多数に対して翻訳のボランティアを募っているため、翻訳の手順や注意点が明文化されていることが多い
\end{itemize}

\hypertarget{ux30abux30c6ux30b4ux30eaux306bux5bfeux3057ux3066ux3069ux306eux3088ux3046ux306aux65b9ux6cd5ux3067ux306aux306bux3092ux660eux3089ux304bux306bux3059ux308bux304b-2}{%
\subsubsection{カテゴリに対してどのような方法でなにを明らかにするか}\label{ux30abux30c6ux30b4ux30eaux306bux5bfeux3057ux3066ux3069ux306eux3088ux3046ux306aux65b9ux6cd5ux3067ux306aux306bux3092ux660eux3089ux304bux306bux3059ux308bux304b-2}}

原文と訳文を比較し、どのような単語が日本語に訳され、どのような単語は訳されずにそのまま使われているのかについて調べる。これにより、プログラミングのチュートリアルにおいて、語彙の訳出の仕方にどのような特徴があるのかを明らかにする

\hypertarget{ux30abux30c6ux30b4ux30eaux306bux5bfeux3057ux3066ux306eux30e1ux30eaux30c3ux30c8ux3068ux30c7ux30e1ux30eaux30c3ux30c8-2}{%
\subsubsection{カテゴリに対してのメリットとデメリット}\label{ux30abux30c6ux30b4ux30eaux306bux5bfeux3057ux3066ux306eux30e1ux30eaux30c3ux30c8ux3068ux30c7ux30e1ux30eaux30c3ux30c8-2}}

\begin{itemize}
\tightlist
\item
  異なる言語で同じ事柄を学ぶときに、原文の学習資源と訳文の学習資源でどのような違いがあるかを比較できる
\item
  単語レベルでの違いは確認できても、文や文書の単位では違いが見つからない可能性がある
\item
  ネット上で公開されている資料が多いため、分析が容易
\item
  英語に慣れることができる
\end{itemize}

\hypertarget{ux30c7ux30fcux30bf-1-2}{%
\subsubsection{データ 1}\label{ux30c7ux30fcux30bf-1-2}}

\href{https://developer.mozilla.org/en-US/docs/Learn}{MDN Learn web
development} と \href{https://developer.mozilla.org/ja/docs/Learn/}{MDN
 ウェブ開発を学ぶ} を比較する

\hypertarget{ux30c7ux30fcux30bfux306eux7279ux5fb4-3}{%
\paragraph{データの特徴}\label{ux30c7ux30fcux30bfux306eux7279ux5fb4-3}}

\begin{itemize}
\tightlist
\item
  Firefox を実装している Mozilla が運営している
\item
  有志が翻訳している
\item
  HTML, CSS, JavaScript
  から、パフォーマンス、ライブラリのことまで、多くのトピックを網羅している
\item
  ウェブアプリケーションを実装するコミュニティのなかで信頼されている情報源である
\item
  HTML で公開されている
\item
  キーワードには説明へのリンクが貼ってある
\end{itemize}

\hypertarget{ux30c7ux30fcux30bfux306bux5bfeux3057ux3066ux3069ux306eux3088ux3046ux306aux65b9ux6cd5ux3067ux306aux306bux3092ux660eux3089ux304bux306bux3059ux308bux304b-3}{%
\paragraph{データに対してどのような方法でなにを明らかにするか}\label{ux30c7ux30fcux30bfux306bux5bfeux3057ux3066ux3069ux306eux3088ux3046ux306aux65b9ux6cd5ux3067ux306aux306bux3092ux660eux3089ux304bux306bux3059ux308bux304b-3}}

ウェブ開発の知識に関する単語で、どのような専門語彙がどのような文字種で訳出されているのかを調べる。\\
これにより、ウェブ開発のチュートリアルにおいて、語彙の訳出の仕方にどのような特徴があるのか、またトピックごとにその分布に違いはあるのかを明らかにする

\hypertarget{ux30c7ux30fcux30bfux306bux5bfeux3057ux3066ux306eux30e1ux30eaux30c3ux30c8ux3068ux30c7ux30e1ux30eaux30c3ux30c8-3}{%
\paragraph{データに対してのメリットとデメリット}\label{ux30c7ux30fcux30bfux306bux5bfeux3057ux3066ux306eux30e1ux30eaux30c3ux30c8ux3068ux30c7ux30e1ux30eaux30c3ux30c8-3}}

\begin{itemize}
\tightlist
\item
  全文テキストが容易に入手できるため、網羅性のある分析が行える
\item
  HTML
  で記述されており、構造化されているため、分析に活用できる項目が多い
\item
  一部翻訳されていない箇所がある
\end{itemize}

\hypertarget{ux30c7ux30fcux30bf-2-1}{%
\subsubsection{データ 2}\label{ux30c7ux30fcux30bf-2-1}}

\href{https://cs50.harvard.edu/x/2021/notes/0/}{cs50 2021
のノート}と\href{https://cs50.jp/x/2021/week0/}{cs50 2021
日本語訳のノート}

\hypertarget{ux30c7ux30fcux30bfux306eux7279ux5fb4-4}{%
\paragraph{データの特徴}\label{ux30c7ux30fcux30bfux306eux7279ux5fb4-4}}

\begin{itemize}
\tightlist
\item
  ハーバード大学の OCW
\item
  Computer Science に関する入門講義であり、例年内容を改訂している
\item
  ノートデータは、見出し、箇条書き、画像、ソースコードで構成されている
\end{itemize}

\hypertarget{ux30c7ux30fcux30bfux306bux5bfeux3057ux3066ux3069ux306eux3088ux3046ux306aux65b9ux6cd5ux3067ux306aux306bux3092ux660eux3089ux304bux306bux3059ux308bux304b-4}{%
\paragraph{データに対してどのような方法でなにを明らかにするか}\label{ux30c7ux30fcux30bfux306bux5bfeux3057ux3066ux3069ux306eux3088ux3046ux306aux65b9ux6cd5ux3067ux306aux306bux3092ux660eux3089ux304bux306bux3059ux308bux304b-4}}

Computer Science
の知識に関する単語で、どのような専門語彙がどのような文字種で訳出されているのかを調べる。\\
これにより、Computer Science
の講義において、語彙の訳出の仕方にどのような特徴があるのか、またトピックごとにその分布に違いはあるのかを明らかにする

\hypertarget{ux30c7ux30fcux30bfux306bux5bfeux3057ux3066ux306eux30e1ux30eaux30c3ux30c8ux3068ux30c7ux30e1ux30eaux30c3ux30c8-4}{%
\paragraph{データに対してのメリットとデメリット}\label{ux30c7ux30fcux30bfux306bux5bfeux3057ux3066ux306eux30e1ux30eaux30c3ux30c8ux3068ux30c7ux30e1ux30eaux30c3ux30c8-4}}

\begin{itemize}
\tightlist
\item
  分析できるデータ量が多い
\item
  HTML でマークアップされている
\item
  Computer Science の入門講義として定評があり、多くの人が利用している
\item
  CS の基礎知識が身につけられる
\end{itemize}

\hypertarget{ux56fdux7acbux56fdux4f1aux56f3ux66f8ux9928ux306eux8535ux66f8ux69cbux6210}{%
\subsection{国立国会図書館の蔵書構成}\label{ux56fdux7acbux56fdux4f1aux56f3ux66f8ux9928ux306eux8535ux66f8ux69cbux6210}}

\hypertarget{ux30abux30c6ux30b4ux30eaux306eux7279ux5fb4-3}{%
\subsubsection{カテゴリの特徴}\label{ux30abux30c6ux30b4ux30eaux306eux7279ux5fb4-3}}

\begin{itemize}
\tightlist
\item
  全国レベルの図書館の蔵書構成である
\item
  国内で発行された資料について、網羅的に収集している
\end{itemize}

\hypertarget{ux30abux30c6ux30b4ux30eaux306bux5bfeux3057ux3066ux3069ux306eux3088ux3046ux306aux65b9ux6cd5ux3067ux306aux306bux3092ux660eux3089ux304bux306bux3059ux308bux304b-3}{%
\subsubsection{カテゴリに対してどのような方法でなにを明らかにするか}\label{ux30abux30c6ux30b4ux30eaux306bux5bfeux3057ux3066ux3069ux306eux3088ux3046ux306aux65b9ux6cd5ux3067ux306aux306bux3092ux660eux3089ux304bux306bux3059ux308bux304b-3}}

蔵書構成に関してメタデータを用いた計量的な分析を行う。\\
これにより、日本の国立図書館が所有する図書の構成を示す。\\
また、国立国会図書館は国内の資料を網羅的に収集しているという特徴を利用して、日本が図書として持っている分野別の知識の総量の指標として考えることと、その他の図書館の蔵書構成の特徴を把握する基準値として考えることができないか検討する。

\hypertarget{ux30abux30c6ux30b4ux30eaux306bux5bfeux3057ux3066ux306eux30e1ux30eaux30c3ux30c8ux3068ux30c7ux30e1ux30eaux30c3ux30c8-3}{%
\subsubsection{カテゴリに対してのメリットとデメリット}\label{ux30abux30c6ux30b4ux30eaux306bux5bfeux3057ux3066ux306eux30e1ux30eaux30c3ux30c8ux3068ux30c7ux30e1ux30eaux30c3ux30c8-3}}

\begin{itemize}
\tightlist
\item
  日本の資料に関して網羅的に収集しているため、日本の図書館の蔵書構成を記述するときの基準値として用いることができる
\item
  国立国会図書館と、大学図書館や公共図書館は性質が異なるため、その違いの内容によっては蔵書構成を比較することに意味がない可能性がある
\end{itemize}

\hypertarget{ux30c7ux30fcux30bf-1-3}{%
\subsubsection{データ 1}\label{ux30c7ux30fcux30bf-1-3}}

国立国会図書館の日本全国書誌番号が割り振られた全書誌の
NDC、NDLC、件名標目、出版年

\hypertarget{ux30c7ux30fcux30bfux306eux7279ux5fb4-5}{%
\paragraph{データの特徴}\label{ux30c7ux30fcux30bfux306eux7279ux5fb4-5}}

\begin{itemize}
\tightlist
\item
  日本全国書誌番号は、法定納本制度に基づき納本された国内出版物と、納本以外の方法(寄贈、購入等)により収集した国内出版物及び外国刊行日本語出版物に対して割り当てられている
\item
  NDC、NDLC、件名標目は図書の内容をもとにした分類である\cite{ChongShi2011}
\end{itemize}

\hypertarget{ux30c7ux30fcux30bfux306bux5bfeux3057ux3066ux3069ux306eux3088ux3046ux306aux65b9ux6cd5ux3067ux306aux306bux3092ux660eux3089ux304bux306bux3059ux308bux304b-5}{%
\paragraph{データに対してどのような方法でなにを明らかにするか}\label{ux30c7ux30fcux30bfux306bux5bfeux3057ux3066ux3069ux306eux3088ux3046ux306aux65b9ux6cd5ux3067ux306aux306bux3092ux660eux3089ux304bux306bux3059ux308bux304b-5}}

図書分類と年代を組み合わせて、それぞれの分類における出版年別の傾向も含めて、国立国会図書館の蔵書構成を示す。\\
その後、それぞれの分類方法ごとの統計について、日本が図書として持っている分野別の知識の総量の指標として考えることと、その他の図書館の蔵書構成の特徴を把握する基準値として考えることができないか検討する。

\hypertarget{ux30c7ux30fcux30bfux306bux5bfeux3057ux3066ux306eux30e1ux30eaux30c3ux30c8ux3068ux30c7ux30e1ux30eaux30c3ux30c8-5}{%
\paragraph{データに対してのメリットとデメリット}\label{ux30c7ux30fcux30bfux306bux5bfeux3057ux3066ux306eux30e1ux30eaux30c3ux30c8ux3068ux30c7ux30e1ux30eaux30c3ux30c8-5}}

\begin{itemize}
\tightlist
\item
  NDC は多くの図書館で使われている
\item
  NDC, NDLC は 1:1 対応でしか紐づけられていない
\item
  国立国会図書館の書誌情報は全件が公開されている
\end{itemize}

\hypertarget{ux52d5ux753bux6559ux6750ux3068ux6587ux5b57ux6559ux6750}{%
\subsection{動画教材と文字教材}\label{ux52d5ux753bux6559ux6750ux3068ux6587ux5b57ux6559ux6750}}

テキスト教材を補完するものとしての動画教材やその逆ではなく、それぞれが同じ知識を記述している対象を扱う

\hypertarget{ux30abux30c6ux30b4ux30eaux306eux7279ux5fb4-4}{%
\subsubsection{カテゴリの特徴}\label{ux30abux30c6ux30b4ux30eaux306eux7279ux5fb4-4}}

\begin{itemize}
\tightlist
\item
  動画教材は、記録された文字、音声、画像、映像による学習教材である
\item
  動画教材は、データ処理技術の向上により、近年多くの場所で用いられるようになっている
\item
  文字教材は、知識を記録する手段として広く使われている
\end{itemize}

\hypertarget{ux30abux30c6ux30b4ux30eaux306bux5bfeux3057ux3066ux3069ux306eux3088ux3046ux306aux65b9ux6cd5ux3067ux306aux306bux3092ux660eux3089ux304bux306bux3059ux308bux304b-4}{%
\subsubsection{カテゴリに対してどのような方法でなにを明らかにするか}\label{ux30abux30c6ux30b4ux30eaux306bux5bfeux3057ux3066ux3069ux306eux3088ux3046ux306aux65b9ux6cd5ux3067ux306aux306bux3092ux660eux3089ux304bux306bux3059ux308bux304b-4}}

その教材に記述されている知識について、テキスト教材と動画教材の説明方法の相違点を探る。具体的には、テキスト教材と動画を順番通りに進めたときの提示される概念の順番や、提示される概念の頻度、そして、特定の概念の説明方法の違いなどを比較する。\\
これらによって、知識の記述方法として特権的な地位をもつ文字教材と比較して、動画による知識の記述がどのような特徴を持つのかを明らかにする。

\hypertarget{ux30abux30c6ux30b4ux30eaux306bux5bfeux3057ux3066ux306eux30e1ux30eaux30c3ux30c8ux3068ux30c7ux30e1ux30eaux30c3ux30c8-4}{%
\subsubsection{カテゴリに対してのメリットとデメリット}\label{ux30abux30c6ux30b4ux30eaux306bux5bfeux3057ux3066ux306eux30e1ux30eaux30c3ux30c8ux3068ux30c7ux30e1ux30eaux30c3ux30c8-4}}

\begin{itemize}
\tightlist
\item
  その教材を通して知識を得ることを目指したものであり、学習者がどのような特徴をもつ教材を通して学んでいるかを明らかにできる
\item
  テキスト、音声、画像、映像など、多様なメディアについて比較検討することができる
  / する必要がある
\item
  動画教材は多くの場所で用いられ始めている
\item
  動画教材についての分析方法はテキスト分析ほど確立されていない
\item
  同じ知識について文字と動画の二種類の学習方法を提供しているデータが少ない
\item
  方法によっては、音声処理や画像処理も身につけられるかもしれない
\end{itemize}

\hypertarget{ux30c7ux30fcux30bf-1-4}{%
\subsubsection{データ 1}\label{ux30c7ux30fcux30bf-1-4}}

\href{https://cs50.harvard.edu/x/2022/}{CS50 2022}

\hypertarget{ux30c7ux30fcux30bfux306eux7279ux5fb4-6}{%
\paragraph{データの特徴}\label{ux30c7ux30fcux30bfux306eux7279ux5fb4-6}}

\begin{itemize}
\tightlist
\item
  ハーバード大学が公開している OCW
\item
  Computer Science に関する入門講義であり、例年内容を改訂している
\item
  動画データとノートデータが公開されている

  \begin{itemize}
  \tightlist
  \item
    他に音声データ・スライドデータ・ソースコード・字幕が公開されているが、動画データに包含される
  \item
    補助教材として、問題が公開されている
  \item
    章によっては、講義形式ではなく事前録画式の動画も公開されている(Shorts)
  \end{itemize}
\item
  動画データとノートデータはそれ自体で学習が可能なものである(ただし、公式ページではノートと講義動画の使い分けについて指示はなく、それ自体で学習可能であると判断したのは佐々木である)
\item
  動画は講義形式で進み、生徒とのやりとりがある
\item
  動画の文字起こしが公開されている
\item
  ノートデータは、見出し、箇条書き、画像、ソースコードで構成されいている
\end{itemize}

\hypertarget{ux30c7ux30fcux30bfux306bux5bfeux3057ux3066ux3069ux306eux3088ux3046ux306aux65b9ux6cd5ux3067ux306aux306bux3092ux660eux3089ux304bux306bux3059ux308bux304b-6}{%
\paragraph{データに対してどのような方法でなにを明らかにするか}\label{ux30c7ux30fcux30bfux306bux5bfeux3057ux3066ux3069ux306eux3088ux3046ux306aux65b9ux6cd5ux3067ux306aux306bux3092ux660eux3089ux304bux306bux3059ux308bux304b-6}}

その教材に記述されている Computer Science
の知識について、ノートと講義動画の説明方法の相違点を探る。具体的には、テキスト教材と動画を順番通りに進めたときの提示される概念の順番や、概念に言及する頻度、そして、特定の概念の説明方法の違いなどを比較する。\\
また、その教材を進めるときに接する語彙数や単語数などを測定し、記述する。\\
これらによって、講義動画とノートの差異を示し、同じ知識について異なる教材で説明するときにどのような差異が発生するのかを明らかにする。

\hypertarget{ux30c7ux30fcux30bfux306bux5bfeux3057ux3066ux306eux30e1ux30eaux30c3ux30c8ux3068ux30c7ux30e1ux30eaux30c3ux30c8-6}{%
\paragraph{データに対してのメリットとデメリット}\label{ux30c7ux30fcux30bfux306bux5bfeux3057ux3066ux306eux30e1ux30eaux30c3ux30c8ux3068ux30c7ux30e1ux30eaux30c3ux30c8-6}}

\begin{itemize}
\tightlist
\item
  分析できるデータ量が多く、文字起こしもされているため扱いやすい
\item
  講義はリアルタイム形式なので、動画教材における特徴の一つである場面転換やエフェクトなどが分析できない
\item
  講師がリアルタイムで話し、学生とのやりとりも行うため、もともと話す予定だった知識と乖離している可能性がある
\item
  ノートと講義動画が独立であるとは明示されていない
\item
  Computer Science の入門講義として定評があり、多くの人が利用している
\item
  CS の基礎知識が身につけられる
\item
  英語に慣れることができる
\end{itemize}

\hypertarget{ux30c7ux30fcux30bf-2-2}{%
\subsubsection{データ 2}\label{ux30c7ux30fcux30bf-2-2}}

\href{https://developer.hashicorp.com/terraform/tutorials}{Terraform
Tutorials}\\
Terraform というウェブサービス管理ツールの公式チュートリアル

\hypertarget{ux30c7ux30fcux30bfux306eux7279ux5fb4-7}{%
\paragraph{データの特徴}\label{ux30c7ux30fcux30bfux306eux7279ux5fb4-7}}

\begin{itemize}
\tightlist
\item
  よく利用されるチュートリアルページは、動画とテキストで構成されている
\item
  動画がないページもある。つまりベースとなる説明はテキストである
\item
  動画があるページは 20\textasciitilde30 程度
\item
  動画は基本的にテキストを読み上げるだけだが、改変している箇所もある
\item
  動画は一つ数分程度である
\item
  文字数で考えると動画のほうが少ない
\item
  動画とテキストはそれ自体で学習が可能なものである(ただし、動画とテキストの使い分けについて指示はなく、それ自体で学習可能であると判断したのは佐々木である)
\item
  テキストのコードブロックをコピーするため、文字教材に依存しているページがある
\end{itemize}

\hypertarget{ux30c7ux30fcux30bfux306bux5bfeux3057ux3066ux3069ux306eux3088ux3046ux306aux65b9ux6cd5ux3067ux306aux306bux3092ux660eux3089ux304bux306bux3059ux308bux304b-7}{%
\paragraph{データに対してどのような方法でなにを明らかにするか}\label{ux30c7ux30fcux30bfux306bux5bfeux3057ux3066ux3069ux306eux3088ux3046ux306aux65b9ux6cd5ux3067ux306aux306bux3092ux660eux3089ux304bux306bux3059ux308bux304b-7}}

動画とテキストの説明を比較し、どのような説明が動画では省かれているかを分析する。また、省かれた説明が動画内で映像などの手段を用いて補完されているかを確認する。\\
これによって、同じ知識を記録するときの動画とテキストの特徴の違いと、それがどのように活用されているかを明らかにする。

\hypertarget{ux30c7ux30fcux30bfux306bux5bfeux3057ux3066ux306eux30e1ux30eaux30c3ux30c8ux3068ux30c7ux30e1ux30eaux30c3ux30c8-7}{%
\paragraph{データに対してのメリットとデメリット}\label{ux30c7ux30fcux30bfux306bux5bfeux3057ux3066ux306eux30e1ux30eaux30c3ux30c8ux3068ux30c7ux30e1ux30eaux30c3ux30c8-7}}

\begin{itemize}
\tightlist
\item
  企業が作成しているチュートリアルで、企業によって使われているソフトウェアである
\item
  ページ単位の対応関係であるので、表現されている知識がコンパクトにまとまっていて解析しやすい
\item
  その教材がどれくらい利用されているかが定かでない(ソフトウェア自体の利用事例などはある程度分かるが)
\item
  表現されている知識が一企業の一製品のものであり、一般性がない
\item
  表現されている知識が特定の業界の特定の職種の人に限定されるものである
\end{itemize}

\hypertarget{ux8eabux306bux3064ux3051ux305fux3044ux30b9ux30adux30ebux306bux3064ux3044ux3066}{%
\section{身につけたいスキルについて}\label{ux8eabux306bux3064ux3051ux305fux3044ux30b9ux30adux30ebux306bux3064ux3044ux3066}}

\begin{itemize}
\tightlist
\item
  PyTorch
  などの機械学習フレームワークを使える技術と自然言語処理に対する理論的な理解
\item
  論文や学術書をストレスなく読めるレベルの英語の語彙・文法知識と英語への慣れ
\item
  分析したい対象に対して、その対象に合った統計的手法を選べるくらいの統計的手法の前提知識
\item
  教育行政、法学、言語学、国際政治、日本近代史、西洋思想史など、何らかの対象が明確な学問分野の基礎知識
\item
  Matplotlib
  などグラフ描画ライブラリの実践的技術と、適切なデータ可視化方法の知識
\end{itemize}

